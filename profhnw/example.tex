\documentclass{profhnw}

\bibliography{example.bib} % Verweis auf Bibliographie

\author{Hans Muster}
\title{Test}

\begin{document}
\maketitle % Titelbild

\section*{Abstract}
\lipsum[1] % Fülltext

\tableofcontents

\section{Einleitung}
\lipsum[2-3] % Fülltext

\section{Hauptteil}
\lipsum[5] 
Quellenangabe mit Biblatex \parencites[S. 203]{Rowling:HarryPotter}.

\subsection{Formeln}

Der Satz von Steiner, siehe \cref{eq:SatzVonSteiner2}, gemäss \citeauthor{Papula:Band1} \parencite{Papula:Band1}.

\begin{align}
\label{eq:SatzVonSteiner2}
J_A &= \frac{m}{12}\cdot l^{2} + m\cdot \left(\frac{l}{2}\right)^{2}
\end{align}

\begin{align}
\label{eq:ROT_X}
A_{x} &= \begin{bmatrix} 
1 & 0 & 0\\ 
0 & \cos{\alpha} & -\sin{\alpha}\\		
0 & \sin{\alpha} & \cos{\alpha} 
\end{bmatrix}
\end{align}

\subsection{Unterkapitel}
\lipsum[4]

\begin{figure}
	\centering
	\includegraphics[width=0.8\linewidth]{FHNW_HT}
	\caption{Bild Floating}
\end{figure}

\begin{center}
	\includegraphics[width=0.6\linewidth]{FHNW_HT}
	\captionof{figure}{Bild ohne Float-Umgebung, wird genau da platziert wo es im Code steht}
\end{center}

\printbibliography


\end{document}